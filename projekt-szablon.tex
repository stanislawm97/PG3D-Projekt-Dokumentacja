\documentclass[10pt,a4paper]{article}
%\documentclass[12pt,a4paper]{article}

\usepackage[utf8]{inputenc}
\usepackage{hyperref}
\usepackage{polski}
\usepackage{graphicx}
\usepackage{listings}
\usepackage{algorithm}
\usepackage{algorithmic}
\usepackage{array}

\usepackage{hyperref}
%\usepackage{antpolt}
\usepackage{amssymb}
\usepackage{multicol}
\usepackage{fancyvrb}

\usepackage{color}

%\setlength{\topskip}{0mm} \setlength{\footskip}{0mm} \setlength{\topmargin}{0mm} \setlength{\marginparwidth}{0mm}
%\setlength{\headsep}{2mm} \setlength{\headheight}{0mm} \setlength{\textheight}{250mm}
%\setlength{\textwidth}{160mm} \setlength{\oddsidemargin}{0mm} \setlength{\evensidemargin}{0mm}

\setlength{\topskip}{0mm} \setlength{\topmargin}{0mm}
\setlength{\oddsidemargin}{0mm} \setlength{\evensidemargin}{0mm}
\setlength{\marginparwidth}{0mm} \setlength{\headsep}{0mm}
\setlength{\headheight}{0mm} \setlength{\textheight}{240mm}
\setlength{\textwidth}{170mm}


\floatname{algorithm}{Algorytm}
\renewcommand{\lstlistlistingname}{Spis listingów}
\renewcommand{\lstlistingname}{Listing}

\newcommand{\centered}[1]{\begin{tabular}{l} #1 \end{tabular}}

\lstset{literate=
  {á}{{\'a}}1 {é}{{\'e}}1 {í}{{\'i}}1 {ó}{{\'o}}1 {ú}{{\'u}}1
  {Á}{{\'A}}1 {É}{{\'E}}1 {Í}{{\'I}}1 {Ó}{{\'O}}1 {Ú}{{\'U}}1
  {à}{{\`a}}1 {è}{{\`e}}1 {ì}{{\`i}}1 {ò}{{\`o}}1 {ù}{{\`u}}1
  {À}{{\`A}}1 {È}{{\'E}}1 {Ì}{{\`I}}1 {Ò}{{\`O}}1 {Ù}{{\`U}}1
  {ä}{{\"a}}1 {ë}{{\"e}}1 {ï}{{\"i}}1 {ö}{{\"o}}1 {ü}{{\"u}}1
  {Ä}{{\"A}}1 {Ë}{{\"E}}1 {Ï}{{\"I}}1 {Ö}{{\"O}}1 {Ü}{{\"U}}1
  {â}{{\^a}}1 {ê}{{\^e}}1 {î}{{\^i}}1 {ô}{{\^o}}1 {û}{{\^u}}1
  {Â}{{\^A}}1 {Ê}{{\^E}}1 {Î}{{\^I}}1 {Ô}{{\^O}}1 {Û}{{\^U}}1
  {Ã}{{\~A}}1 {ã}{{\~a}}1 {Õ}{{\~O}}1 {õ}{{\~o}}1
  {œ}{{\oe}}1 {Œ}{{\OE}}1 {æ}{{\ae}}1 {Æ}{{\AE}}1 {ß}{{\ss}}1
  {ű}{{\H{u}}}1 {Ű}{{\H{U}}}1 {ő}{{\H{o}}}1 {Ő}{{\H{O}}}1
  {ç}{{\c c}}1 {Ç}{{\c C}}1 {ø}{{\o}}1 {å}{{\r a}}1 {Å}{{\r A}}1
  {€}{{\euro}}1 {£}{{\pounds}}1 {«}{{\guillemotleft}}1
  {»}{{\guillemotright}}1 {ñ}{{\~n}}1 {Ñ}{{\~N}}1 {¿}{{?`}}1
  {ł}{\l{}}1 {ś}{{\'s}}1 {ź}{{\.z}}1  {ó}{{\'o}}1
  {ą}{{\k{a}}}1
}

\begin{document}

\pagestyle{plain}
\begin{center}
\begin{center}
    \includegraphics[width=.55\textwidth]{logo}
\end{center}
\vspace{0.5cm}
\textsc{\Huge{Uniwersytet Zielonogórski}}\\
\LARGE{Wydział Informatyki, Elektrotechniki i~Automatyki}\\
\large{Instytut Sterowania i Systemów Informatycznych}\\
\vspace{0.5cm}
\Large{Programowanie gier 3D -- Projekt}\\
Prowadzący: Mgr inż. Marcin Skobel \\ 
\vspace{2cm}
\LARGE{Wściekły Maks}\\
\vspace{2cm} 
\Large{Stanisław Mól, Erwin Konkel} \\
\Large{Grupa dziekańska: 33INF-SSI-SP} \\
\vspace{0.5cm} 
\Large{Data oddania projektu: 22.01.2021}
\vspace{4cm}
\begin{flushleft}
	Ocena: ..........................................
\end{flushleft}
\vspace{1cm}
\end{center}

\footnotesize
\tableofcontents

\footnotesize
\lstlistoflistings

\noindent\makebox[\linewidth]{\rule{0.6\paperwidth}{0.4pt}}

\clearpage
\section{Wprowadzenie}
charakterystyka gry, gatunek, platforma sprzętowa, wprowadzenie do rozgrywki

\clearpage
\section{Charakterystyka techniczna}
statystyki techniczne (np. liczba klatek na sekundę, obciążenie pamięci, rozmiar na dysku itp.), najciekawsze rozwiązania techniczne zastosowane w grze (skrypty, komponenty itp.)\\

\clearpage
\section{Scenariusz gry}

\clearpage
\section{Rozgrywka}

\clearpage
\section{Opis wkładu własnego w realizację projektu}

\vspace{0.5cm}
\begin{description}
  \item[Stanisław Mól:] \hfill
  	\begin{itemize}
 	  \item Postać gracza
 	  \item Poruszanie się gracza
	  \item Pierwszy przeciwnik
	  \item Teren poruszania się przeciwnika
	  \item System animacji przeciwnika
	  \item Atakowanie / poruszanie się przeciwnika (pojawiły się błędy)
	  \item Naprawa błędów
	  \item Zmiana przeciwnika na Humanoidalnego
	  \item Dodanie MainMenu
	\end{itemize}
  \item[ Erwin Konkel:] \hfill
  	\begin{itemize}
          \item Skybox
  	  \item Design mapy
 	  \item Tekstury
  	  \item Wieże
	  \item Wioska początkowa
 	  \item Granice mapy
  	  \item Ulepszenie ścieżki 
  	  \item Poprawione oświetlenie
	  \item Rozwijanie wioski
	  \item NPC
	  \item Dodanie nowego poziomu (nowa wioska)
	\end{itemize}
\end{description}

\clearpage
\section{Spis zastosowanych assetów z krótką charakterystykę}
\begin{itemize}
  \item Third Person Controller - Basic Locomotion FREE - https://assetstore.unity.com/packages/tools/utilities/third-person-controller-basic-locomotion-free-82048
  \item Free Low Poly Desert Pack - https://assetstore.unity.com/packages/3d/environments/free-low-poly-desert-pack-106709
  \item Fantasy Skybox Free - https://assetstore.unity.com/packages/2d/textures-materials/sky/fantasy-skybox-free-18353
  \item Ruined Tower Free - https://assetstore.unity.com/packages/3d/environments/ruined-tower-free-66495
  \item Spider Green - https://assetstore.unity.com/packages/3d/characters/animals/insects/spider-green-11869
  \item Campfire Pack - https://assetstore.unity.com/packages/3d/environments/fantasy/campfire-pack-11256
\end{itemize}

\clearpage
\section{Podsumowanie}
opisanie trudności oraz rozwiązania, wskazanie pozytywnych cech projektu oraz ewentualnych planów i szans na dalszy rozwój projektu.


\noindent\makebox[\linewidth]{\rule{0.6\paperwidth}{0.4pt}}
\begin{center}
	Koniec dokumentu.
\end{center}
\end{document}
